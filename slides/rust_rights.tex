\documentclass[lualatex,utf8]{beamer}
\usepackage{graphicx}
\usepackage{amssymb}
\usepackage{luacode}
\usepackage{colortbl}
\usepackage{booktabs}
\usepackage{tabu}
\usepackage{minted}

\setbeameroption{show notes}
%\setbeameroption{hide notes}
\usetheme{Madrid}
\setminted{autogobble,fontsize=\footnotesize}%fontfamily=Inconsolata,

\title[Embedded Rust]{Rust in Embedded Systems}
  \author{Taoda}
  \institute{YITU tech}
  \date{Mar\ 2019}

\renewcommand{\emph}{\textbf}

\begin{document}

\begin{frame}
\titlepage
\end{frame}

\begin{frame}
  \frametitle{Outline}
  \tableofcontents
\end{frame}


\section{Why Rust Is A Candidate for Embedded Systems}

\AtBeginSection[]
{
\begin{frame}<beamer>
  \frametitle{Outline}
  \tableofcontents[currentsection,subsubsectionstyle=show/show/show/shaded]
\end{frame}
}

\begin{frame}
  \frametitle{Why Rust Is Suitable for Embedded Systems}
  \begin{block}{On embedded systems}
    \begin{tabu}{*4{X[c]}}
      \rowcolor{black!40}
      & CA320 & N-Box & H-Box \\
      \taburowcolors 2{black!10 .. black!10}
      CPU & Hi3516av200 & Nvidia TX2 & Hi3559a \\
      \taburowcolors 2{black!20 .. black!20}
      RAM & 1GB & 8GB & 4GB \\
      \taburowcolors 2{black!10 .. black!10}
      Flash & 16MB & 32GB & 16GB \\
    \end{tabu}
  \end{block}
\end{frame}

\begin{frame}
  \frametitle{Why Rust Is Suitable for Embedded Systems}
  \begin{block}{My requirements}
    \begin{enumerate}
    \item \emph{ARM support}
    \item \emph{Linux support}
    \item JiTeng works on C/C++, so, \emph{easy interoperability to C/C++}
    \item I work on ``business level'', so, I care about both \emph{productivity} and \emph{safety}.
      \begin{itemize}
      \item e.g., how easy to write multi-threaded HTTP servers and clients without fear.
      \item I don't want to gdb core dumps in the field.
      \end{itemize}
    \item High \emph{performance}, low \emph{memory} footprint and \emph{disk} usage.
      \begin{itemize}
      \item GC free
      \end{itemize}
    \end{enumerate}
  \end{block}
\end{frame}

\begin{frame}
  \frametitle{Why Rust Is Suitable for Embedded Systems}
  \begin{exampleblock}{Candidates}
    \begin{itemize}
    \item Python/Ruby/NodeJS/PHP/Erlang, out! (performance)
    \item All languages on JVM, out! (GC \& disk usage \& ARM support)
    \end{itemize}
    I want a ``real'' system's language.
  \end{exampleblock}
\end{frame}

\begin{frame}
  \frametitle{Why Rust Is Suitable for Embedded Systems}
  \begin{block}{Candidates}
    \begin{tabu}{*4{X[c]}}
      \rowcolor{black!40}
      Language & TIOBE-201902 & GitHub-18 & GitHub Growth 18' \\
      \taburowcolors 2{black!10 .. black!10}
      C & 2 & 9 & \\
      \taburowcolors 2{black!20 .. black!20}
      C++ & 4 & 5 & \\
      \taburowcolors 2{black!10 .. black!10}
      Assembly & 11 & & \\
      \taburowcolors 2{black!20 .. black!20}
      Go & 18 & & 7(1.5x)\\
      \taburowcolors 2{black!10 .. black!10}
      Rust & 37 & & 5(1.7x)\\
    \end{tabu}
  \end{block}
\end{frame}

\section{Language Battles}
\subsection{On ARM support}

\begin{frame}
  \frametitle{On ARM support \& interoperability to C/C++}
  \begin{block}{Workflow about compiling a Rust program}
    \begin{center}
      \includegraphics{fig/rust_workflow.pdf}
    \end{center}
    \begin{itemize}
    \item ARM support: as good as Clang
    \item Interoperability to C/C++: binary compatible, extremely thin FFI
    \end{itemize}
  \end{block}
\end{frame}

\subsection{On interoperability to C/C++}

\begin{frame}[fragile]
  \frametitle{On interoperability to C/C++}
  \begin{exampleblock}{A function in C}
    \begin{minted}{rust}
      snappy_status snappy_compress(const char* input,
                                    size_t input_length,
                                    char* compressed,
                                    size_t* compressed_length);
    \end{minted}
  \end{exampleblock}
  \begin{block}{Counterpart in Rust}
    \begin{minted}{rust}
      extern crate libc;
      use libc::{c_int, size_t};

      #[link(name = "snappy")]
      extern {
          fn snappy_compress(input: *const u8,
                             input_length: size_t,
                             compressed: *mut u8,
                             compressed_length: *mut size_t) -> c_int;
      }
    \end{minted}
  \end{block}
\end{frame}

\subsection{On performance \& memory footprint}

\begin{frame}
  \frametitle{On performance \& memory footprint}
  \begin{block}{The Computer Language Benchmarks Game}
    \begin{center}
      \includegraphics{fig/benchmark.pdf}
    \end{center}
  \end{block}
\end{frame}

\begin{frame}
  \frametitle{On performance \& memory footprint}
  \begin{block}{Conclusions}
    \begin{itemize}
    \item Rust runs as fast as C/C++, 2-4x faster than Go.
    \item Rust costs as many memory as C/C++, 2-4x smaller than Go.
    \end{itemize}
  \end{block}
\end{frame}

\subsection{On safety}

\begin{frame}
  \frametitle{On safety}
  \begin{block}{Lessons learnt from C++}
    \begin{itemize}
    \item Object ownership (memory leak \& double-free)
    \item Object lifetime (dangling pointer dereference)
    \item Dangerous pointer arithmetic (buffer overflow)
    \item Multiple threads access a same object, and at least one of them writes. (race condition)
    \end{itemize}
  \end{block}
  \begin{exampleblock}{The C++ way}
    \begin{itemize}
    \item Language provides tools, such as move semantics, std::unique\_ptr, ...
    \item Programmers wisely adopt these tools as rules of discipline.
    \end{itemize}
  \end{exampleblock}
\end{frame}

\begin{frame}[fragile]
  \frametitle{On safety}
  \begin{alertblock}{a leaky example}
    \begin{minted}{cpp}
      void funcA(T&& obj)
      {
          std::unique_ptr<T> aDeleter(new T(std::move(obj)));
          ThreadPool::TaskPtr pHandlingTask;
          pHandlingTask = boost::make_shared<ThreadPool::Task>(
              boost::bind(&funcB, aDeleter.get()));
          if (...) {
              mThreadPool.Submit(pHandlingTask);
          }
          aDeleter.release();
      }
    \end{minted}
  \end{alertblock}
\end{frame}

\begin{frame}
  \frametitle{On safety}
  \begin{block}{The Rust way}
    \begin{itemize}
    \item Ownership
    \item Borrowing checking
    \item Lifetime
    \end{itemize}
  \end{block}
\end{frame}

\begin{frame}[fragile]
  \frametitle{On safety}
  \begin{alertblock}{Ownership}
    \begin{minted}[frame=single]{rust}
      let s1 = String::from("hello");
      let s2 = s1;

      println!("{}, world!", s1);
    \end{minted}
    \begin{minted}[frame=single]{text}
      error[E0382]: use of moved value: `s1`
       --> src/main.rs:5:28
        |
      3 |     let s2 = s1;
        |         -- value moved here
      4 |
      5 |     println!("{}, world!", s1);
        |                            ^^ value used here after move
        |
        = note: move occurs because `s1` has type `std::string::String`, which
        does not implement the `Copy` trait
    \end{minted}
  \end{alertblock}
\end{frame}

\begin{frame}[fragile]
  \frametitle{On safety}
  \begin{block}{Rule of Borrowing}
    At any time, you can have, to a same object,
    \begin{itemize}
    \item either one mutable reference
    \item or any number of immutable references
    \end{itemize}
    exclusively.
  \end{block}
  \begin{alertblock}{~}
    \begin{minted}{rust}
      let mut s = String::from("hello");
      let r1 = &s; // no problem
      let r2 = &s; // no problem
      let r3 = &mut s; // BIG PROBLEM
      println!("{}, {}, and {}", r1, r2, r3);
    \end{minted}
  \end{alertblock}
\end{frame}

\begin{frame}[fragile]
  \frametitle{On safety}
  \begin{block}{Lifetime}
    An intuitive idea: any reference must live shorter than the lifetime of the object to which it references.
  \end{block}
  \begin{alertblock}{~}
    \begin{minted}{rust}
      fn f() -> &str {
        let result = String::from("really long string");
        result.as_str()
      }
    \end{minted}
  \end{alertblock}
\end{frame}

\begin{frame}[fragile]
  \frametitle{On safety}
  \begin{alertblock}{But lifetime is a little bit complex}
    \begin{minted}[frame=single]{rust}
      fn longest(x: &str, y: &str) -> &str {
        if x.len() > y.len() {
          x
        } else {
          y
        }
      }
    \end{minted}
    \begin{minted}[frame=single]{text}
      error[E0106]: missing lifetime specifier
       --> src/main.rs:1:33
        |
      1 | fn longest(x: &str, y: &str) -> &str {
        |                                 ^ expected lifetime parameter
        |
        = help: this function's return type contains a borrowed value, but
        the signature does not say whether it is borrowed from `x` or `y`
    \end{minted}
  \end{alertblock}
\end{frame}

\begin{frame}[fragile]
  \frametitle{On safety}
  \begin{exampleblock}{Lifetime annotation}
    \begin{minted}[frame=single]{rust}
      fn longest<'a>(x: &'a str, y: &'a str) -> &'a str {
        if x.len() > y.len() {
          x
        } else {
          y
        }
      }
    \end{minted}
    This function signature now tells Rust that for some lifetime \mintinline{rust}|'a|,
    \begin{itemize}
    \item both of parameters live at least as long as lifetime \mintinline{rust}|'a|;
    \item also the returned string slice will live at least as long as lifetime \mintinline{rust}|'a|.
    \end{itemize}
  \end{exampleblock}
\end{frame}

\subsection{On productivity}

\begin{frame}
  \frametitle{On productivity}
  \begin{block}{What is productivity in software development?}
    My opinions: the speed of delivering reliable features.
    \begin{enumerate}
    \item Avoid bugs in coding $>$ Find bugs in testing $>$ Fixes bugs in the fields.
      \begin{itemize}
      \item Which implies, safety really matters.
      \end{itemize}
    \item Modeling the nature in a nature way.
    \end{enumerate}
  \end{block}
\end{frame}

\begin{frame}
  \frametitle{On productivity}
  \begin{block}{Comparisons on code lines}
    For a TCP echo server:
    \begin{itemize}
    \item Rust +tokio: 22 LoC
    \item C++ +boost::asio: 89 LoC
    \item Java +netty: 48 LoC
    \item Python3 (asyncio): 19 LoC
    \item Go: 39 LoC
    \end{itemize}
  \end{block}
\end{frame}

\begin{frame}
  \frametitle{On productivity}
  \begin{block}{OOP}
    OOP tries to shot two birds by one bullet:
    \begin{itemize}
    \item as a tool for modeling
    \item as a tool for reusing codes
    \end{itemize}
  \end{block}
\end{frame}

\begin{frame}
  \frametitle{On productivity}
  \begin{block}{Platonism OOP}
    \begin{itemize}
    \item a.k.a., subclassing
    \item for modeling:
      \begin{itemize}
      \item Ducks are birds. Every bird can fly. So, ducks can fly.
      \item What about a bird who cannot fly, e.g., ostrichs?
      \end{itemize}
    \item for reusing code
      \begin{itemize}
      \item diamond problem
      \end{itemize}
    \end{itemize}
  \end{block}
\end{frame}

\begin{frame}
  \frametitle{On productivity}
  \begin{block}{Duck-typing OOP}
    \begin{itemize}
    \item a.k.a., interface-oriented programming
    \item for modeling:
      \begin{itemize}
      \item Lib authors always doubt some interfaces are missing to declare.
      \end{itemize}
    \item for reusing code: no way
    \end{itemize}
  \end{block}
\end{frame}

\begin{frame}[fragile]
  \frametitle{On productivity}
  \begin{block}{Trait impl in rust}
    \begin{minted}{rust}
      fn main() {
        5.f();
      }
      pub trait T {
        fn f(&self) -> ();
      }
      impl T for u64 {
        fn f(&self) -> () {
          println!("{}", self);
        }
      }
    \end{minted}
  \end{block}
\end{frame}

\begin{frame}[fragile]
  \frametitle{On productivity: FP paradigm}
  \begin{columns}[t]
    \begin{column}{.45\textwidth}
      \begin{block}{flawed code in Java}
        \begin{minted}{java}
          return a.f().g().h();
        \end{minted}
      \end{block}
    \end{column}
    \begin{column}{.45\textwidth}
      \begin{block}<2>{fixed code in Java}
        \begin{minted}{java}
          B b = null;
          if (a != null) {
            b = a.f();
          }
          C c = null;
          if (b != null) {
            c = b.g();
          }
          D d = null;
          if (c != null) {
            d = c.h();
          }
          return d;
        \end{minted}
      \end{block}
    \end{column}
  \end{columns}
\end{frame}

\begin{frame}[fragile]
  \frametitle{On productivity: FP paradigm}
  \begin{block}{FP way in Rust}
    \begin{columns}
      \begin{column}[t]{.45\textwidth}
        \begin{minted}{rust}
          struct A {}
          impl A {
              fn f(&self) -> Option<B> {...}
          }
          struct B {}
          impl B {
              fn g(&self) -> Option<C> {...}
          }
          struct C {}
          impl C {
              fn h(&self) -> Option<D> {...}
          }
          struct D {}
        \end{minted}
      \end{column}
      \begin{column}[t]{.45\textwidth}
        \begin{minted}{rust}
          let d = Some(a)
              .and_then(|x| x.f())
              .and_then(|x| x.g())
              .and_then(|x| x.h());
        \end{minted}
      \end{column}
    \end{columns}
  \end{block}
\end{frame}

\begin{frame}
  \frametitle{On productivity: meta programming}
  \begin{block}{~}
    \begin{itemize}
    \item C macro is well-known dangerous.
    \item C++ templating is strong but complex.
    \item Rust macro is not so expressive but also much simpler.
    \end{itemize}
  \end{block}
\end{frame}

\begin{frame}[fragile]
  \frametitle{On productivity: meta programming}
  \begin{block}{Rust macro example}
    \begin{minted}{rust}
      macro_rules! vec {
        ( $( $x:expr ),* ) => {
          {
            let mut temp_vec = Vec::new();
            $(
              temp_vec.push($x);
            )*
            temp_vec
          }
        };
      }

      let v: Vec<u32> = vec![1, 2, 3];
    \end{minted}
  \end{block}
\end{frame}

\subsection{On learning curve \& labor supply}

\begin{frame}
  \frametitle{On labor supply}
  \begin{block}{In the labor market,}
    \begin{itemize}
    \item Fact: there are few supply of Rust programmers.
    \item Fact: there are a few supply of qualified C++ programmers.
    \item Opinion: there is no easy way to distinguish qualified and unqualified C++ programmers.
    \end{itemize}
  \end{block}
\end{frame}

\begin{frame}
  \frametitle{On learning curve}
  \begin{block}{My experience}
    From newbie to being familiar with toy project experience,\\
    which means, being able to work on real projects with google.
    \begin{tabu}{*4{X[c]}}
        \rowcolor{black!40}
        Language & Learning & *MPost & testa \\
        \taburowcolors 2{black!10 .. black!10}
        C & 1 months & -- & -- \\
        \taburowcolors 2{black!20 .. black!20}
        Java & 1 months & 3 weeks & --\\
        \taburowcolors 2{black!10 .. black!10}
        C++ & 1 week & -- & 3 days\\
        \taburowcolors 2{black!20 .. black!20}
        Python & 3 days & 1 month & --\\
        \taburowcolors 2{black!10 .. black!10}
        Scala & 2 weeks & -- & --\\
        \taburowcolors 2{black!20 .. black!20}
        Erlang & 2 weeks & -- & 1 weeks\\
        \taburowcolors 2{black!10 .. black!10}
        Go & 1 day & -- & --\\
        \taburowcolors 2{black!20 .. black!20}
        Clojure & 1 months & -- & 1 week\\
        \taburowcolors 2{black!10 .. black!10}
        Lua & 3 days & 1 weeks & 1 night\\
        \taburowcolors 2{black!20 .. black!20}
        Rust & 1 week & -- & 2 nights\\
    \end{tabu}
  \end{block}
\end{frame}

\begin{frame}
  \frametitle{On learning curve}
  \begin{block}{What helps me in learning Rust?}
    \begin{itemize}
    \item system knowledges: compiling\&linking, stack\&heap, multithreading, async io
    \item PL knowledges: OOP, FP
    \end{itemize}
    Which of the above doesn't a qualified C++ programmer need to know?
  \end{block}
\end{frame}

\section{Conclusions}

\begin{frame}
  \frametitle{Conclusions}
  \begin{block}{What I didn't say}
    \begin{itemize}
    \item JiTeng should engage Rust
    \item server-app developers should engage Rust
    \end{itemize}
  \end{block}
  \begin{block}{What I did say}
    \begin{itemize}
    \item Rust is a better C++.
    \item If you are a ``business level'' developer and you care about CPU/memory, try Rust.
    \end{itemize}
  \end{block}
\end{frame}

\begin{frame}
  \frametitle{Adopted Companies}
  \noindent
  \includegraphics[width=\textwidth]{fig/sensetime.png}
\end{frame}

\begin{frame}
  \begin{center}
    \Huge
    Thanks
  \end{center}
\end{frame}

\end{document}
